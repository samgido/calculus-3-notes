\author{Samuel Gido}
\title{Exam 3 Notes}
\date{Fall Semester 2023}

\documentclass[12pt, letterpaper]{article}

\usepackage{fullpage}
\usepackage[bottom]{footmisc}
\usepackage{setspace}
\onehalfspacing

\usepackage{mathtools}
\usepackage{xcolor}
\usepackage{amsfonts}

\begin{document}
\maketitle

\section*{Theorems}

  \subsection*{Fubini's Theorem}

  Given  \( R := \{ (x, y)\,: \,a \leq x \leq b,\ c \leq y \leq d \}\)

  \begin{equation*}
    \iint\limits_R f(x,y)\, dy\,dx \\
    =
    \int_{a}^{b} \int_{c}^{d} f(x,y)\, dy\, dx \\
    = 
    \int_{c}^{d} \int_{a}^{b} f(x, y)\, dx\, dy \\
  \end{equation*}

\pagebreak

\section*{General Integration}

  {\Large
    For some function f, and some region \(\Omega\):
  }

  \begin{equation*}
    \int \limits_{\Omega} f\, d\\(something) = I
  \end{equation*}

  {\Large
    $I$ is the average value of $f$ on $\Omega$ times the size\footnote{The meaning of size differs based on the region you're integrating over} of $\Omega$
  }

\pagebreak

\section*{Integration Over a 2D Region}

  2-dimensional regions are be described by how well they can be ''sliced'' \newline
  
  Type 1 regions are described by the following, given:

  \[
    D_{1} := \{(x,y) : a \leq x \leq b,\, f_{1}(x) \leq y \leq f_{2}(x)\} \\ 
  \]

  \begin{equation*}
    \int \limits_{D_{1}} h(x,y)\,dy\,dx  \\
  \end{equation*}

  And Type 2 regions can be described similarly: 
  
  \[
    D_{2} := \{(x,y) : g_{1}(y) \leq x \leq g_{2}(y),\, c \leq y \leq d \}
  \]

  \begin{equation*}
    \int \limits_{D_{2}} h(x,y)\,dx\,dy \\ 
  \end{equation*}

  Clearly, these integrals are setup with the best order of integration

\pagebreak

  \section*{Coordinate Systems}

  4 systems are used in this course:

  \begin{enumerate}
    \item Cartiesian: 2D \& 3D
    \item \textcolor{cyan}{Polar: 2D}
    \item \textcolor{orange}{Cylindrical: 3D}
    \item \textcolor{teal}{Spherical: 3D}
  \end{enumerate}

  \subsection*{Conversions}

  $x \rightarrow \textcolor{cyan}{rcos(\theta)} \rightarrow \textcolor{orange}{rcos(\theta)} \rightarrow \textcolor{teal}{\rho\sin(\theta)cos(\phi)}$ \newline
  $y \rightarrow \textcolor{cyan}{rsin(\theta)} \rightarrow \textcolor{orange}{rsin(\theta)} \rightarrow \textcolor{teal}{\rho\sin(\theta)sin(\phi)}$ \newline
  $z \rightarrow \textcolor{orange}{z} \rightarrow \textcolor{teal}{\rho\cos(\phi)}$ \newline

  \noindent $\textcolor{teal}{\rho} \rightarrow \sqrt{x^{2} + y^{2} + z^{2}} \rightarrow \textcolor{orange}{\sqrt{r^{2} + z^{2}}}$ \newline
  $\textcolor{teal}{\theta} \rightarrow arctan(\frac{y}{x})$ \newline
  $\phi \rightarrow arccos(\frac{z}{\rho}) \rightarrow arccos(\frac{z}{\rho})$ \newline

  \subsection*{Jacobian constants}

  $dA \rightarrow \textcolor{cyan}{r\,dr\,d\theta}$ \newline
  $dV \rightarrow \textcolor{orange}{r\,dr\,d\theta\,dz}$ \newline
  $dV \rightarrow \textcolor{teal}{\rho^{2}sin(\phi)\,d\rho\,d\theta\,d\phi}$

\pagebreak

  \section*{Vector Fields}
  Considering $ \vec{F} := <x, y, z>$

  \subsection*{Divergence of $\vec{F}$}

  - Denoted by $\nabla \cdot \vec{F} = F_{x}^{1}\,+\,F_{y}^{2}\,+\,F_{z}^{2}$ \newline
  - Measures how much $\vec{F}$ is ''spreading out''
  - $\nabla \cdot \vec{F} = 0$ when $\vec{F}$ is \textbf{incompressable}
  
  \subsection*{Curl of $\vec{F}$}
  - Denoted by $\nabla \times \vec{F} = \vec{F} \, \times <\partial_x ,\, \partial_y ,\, \partial_z >$ \newline
  - $\nabla \times \vec{F} = \vec{0}$ when $\vec{F}$ is \textbf{irrotational and conservative} \newline
  - Magnitude of $\nabla \times \vec{F}$ measures angular velocity \newline
  - Direction of $\nabla \times \vec{F}$ measures axis of rotation by right hand rule \newline

  \subsection*{Conservative Vector Fields}
  
  - \(\vec{F}\) is conservative if there exists some function, $f$\footnote{Called the potential function}, where $\nabla f = \vec{F}$ \newline

  \noindent - Full Theorems: 

  \begin{center}
    if \(\vec{F} : \mathbb{R}^{2} \rightarrow \mathbb{R}^{2}\) is defined as differentiable on a simply connected domain \\
    and \\
    \(F_{y}^{1} = F_{x}^{2}\) \\
    then \(\vec{F}\) is conservative.
  \end{center}

  \begin{center} 
    if \(\vec{F} : \mathbb{R}^{2} \rightarrow \mathbb{R}^{2}\) is defined as differentiable on a simply connected domain \\
    and \\ 
    \(\nabla \times \vec{F} = 0\) \\
    then \(\vec{F}\) is conservative.
  \end{center}

\pagebreak

  \section*{Line Integrals}

  \subsection*{Scalar line integrals}
  Let \(f\) be a function with a domain that includes a smooth curve \(C\) that is parametrized by \(r(t) = <x(t),\, y(t),\, z(t)>,\, a \leq t \leq b\).\\ 

  The scalar integral of \(f\) along \(C\) is 
  \begin{center}
    \(\int \limits_{C} f(x,y,z)\,ds = \int_{a}^{b}f(\vec{r}(t)) \cdot \|\vec{r}'(t)\|\,dt\)
  \end{center}

  \subsection*{Vector line integrals}
  The vector line integral of a vector field \(\vec{F}\) along oriented smooth curve \(C\) is 
  \begin{center}
    \( \int \limits_C \vec{F} \cdot \vec{T}\,ds \) \\
    Which can be simplified to \\
    \( \vec{F} \cdot \vec{T}\,ds = \vec{F}(\vec{r}(t)) \cdot \frac{\vec{r}\,'(t)}{\|\vec{r}\,'(t)\|} \cdot \vec{r}\,'(t)\,dt = \vec{F}(\vec{r}(t)) \cdot \vec{r}\,'(t)\,dt\) \\
    Therefore the vector line integral of \(\vec{F}\) over \(C\) is \\
    \( \int_{a}^{b} \vec{F}(\vec{r}(t)) \cdot \vec{r}\,'(t)\,dt \)
  \end{center}

  \subsection*{Orientation of line integrals}
  Orientation and parametrization \textbf{do not matter for scalar line integrals}. \\

  \noindent However, they do matter for vector line integrals. This isn't a huge problem, just remember

  \begin{center}
    \(\int_{a}^{b}\vec{F}\,d\vec{r} = (-1)\int_{b}^{a} \vec{F}\,  d\vec{r} \)
  \end{center}

\pagebreak

  \subsection*{Fundamental theorem of line integrals}
  When calculating a vector line integral with a conservative vector field, the fundamental theorem of line integrals applies: 

  \begin{center}
    \(\int \limits_C \vec{F} \, d\vec{r} = f(b) - f(a)\) \\
    or \\
    \(\int \limits_C \nabla f \, d\vec{r} = f(b) - f(a)\) 
  \end{center}

  \subsection*{Line integrals with respect to x, y, z}
  Assuming \(\vec{r}(t) = <x(t),\,y(t),\,z(t)>\)

  \begin{center}
    \(\int \limits_C f \, dx = \int_{a}^{b}f(\vec{r}(t)) \cdot x'(t)\,dt\) \\
  \end{center}

\pagebreak

\section*{Integration of functions over surfaces}

  \subsection*{Parametric Surfaces}
  Like with line integrals, when doing a surface integral the surface has to be parametrized. 
  Parametrized surfaces have this form:

  \begin{center}
    \(r(u, v) = <x(u, v),\,y(u,v),\,z(u, v)>\)
  \end{center}

  \noindent With this form, the parameter domain is a set of 2 points on the \(uv\) plane.

  \subsection*{Surface Integral of a Scalar Function}

  Assume \(S\) is a piecewise smooth surface with parametrization \(r(u, v) = <x(u, v),\,y(u,v),\,z(u,v)>\) and parameter domain \(D\). 
  The surface integral of a scalar-valued function \(f\) over \(S\) is 
  \begin{center}
    \(\iint \limits_S f(x,y,z)\,dS = \iint \limits_D f(r(u,v)) \cdot \|r_{u} \times r_{v{}}\|\,dA\)
  \end{center}
  
  \noindent This somewhat resembles the definition for a line integral, with one key difference being the term \(\|r'(t)\|\) in the line integral has become \(\|r_{u}\times r_{v}\|\). \(r'(t)\) is tangent to the curve whereas the vector \(r_{u} \times r_{v}\) is perpendicular to the surface

\end{document}