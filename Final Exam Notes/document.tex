\author{Samuel Gido}
\title{Final Exam Notes}
\date{Fall Semester 2023}

\documentclass[12pt, letterpaper]{article}

\usepackage{fullpage}
\usepackage[bottom]{footmisc}
\usepackage{setspace}
\onehalfspacing

\usepackage{mathtools}
\usepackage{xcolor}
\usepackage{amsfonts}

\begin{document}
  \maketitle
  \section*{Divergence Theorem}

  The diverence theorem relates a flux integral of vector field \(F\) over a closed surface \(S\) to a triple integral of the divergence of \(F\) over the solid that \(S\) encloses \\

  \noindent In 2 dimensions, also the flux form of Green's theorem, divergence theorem is
  \begin{center}
    \(\iint \limits_T \nabla \cdot \vec{F}\,dV = \int \limits_{\partial T} \vec{F} \cdot \vec{n} \, ds\)
  \end{center}

  \noindent And in 3 dimensions, the actual divergence theorem
  \begin{center}
    \(\iiint \limits_R \nabla \cdot \vec{F}\,dV = \iint \limits_{\partial R}\vec{F} \cdot \vec{n}\,dS\)
    
  \end{center}

  \noindent In this scenario, the \(\partial\) symbol next to a set denotes the boundaries of a set. Also, the normal vectors represent the outward orientation of the unit vector.

\pagebreak

\section*{Stoke's Theorem}

Definition: \\
The flux of curl \(F\) across a surface \(S\) can be found with only information about the values of \(F\) along the boundary of \(S\). Also, we can calculate the line integral of vector field \(F\) along the boundary of the surface \(S\) by translating to a double integral of the curl of \(F\) over \(S\)

\begin{center}
  \(\iint \limits_S (\nabla \times \vec{F}) \cdot \vec{n} \, dS = \int \limits_{\partial S} \vec{F} \cdot d\vec{r}\)
\end{center}

\end{document}